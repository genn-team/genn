\hypertarget{Quickstart_page_quick_sec}{}\section{Quickstart}\label{Quickstart_page_quick_sec}
Ge\+N\+N is based on the idea of code generation for the involved G\+P\+U or C\+P\+U simulation code for neuronal network models but leaves a lot of freedom how to use the generated code in the final application. To facilitate the use of Ge\+N\+N on the background of this philosophy it comes with a number of complete examples containing both the model description code that is used by Ge\+N\+N for code generation and the \char`\"{}user side code\char`\"{} to run the generated model and safe the results. Running these complete examples should be achievable in a few minutes. The necessary steps are described below.\hypertarget{Quickstart_page_unix_quick}{}\subsection{Unix Quickstart}\label{Quickstart_page_unix_quick}
In order to get a quick start and run a provided model, open a shell, navigate to {\ttfamily Ge\+N\+N/tools} and type 
\begin{DoxyCode}
make
\end{DoxyCode}
 This will compile additional tools for creating and running example projects. For a first complete test, the system is best used with a full driver program such as in the \hyperlink{Examples_ex_mbody}{Insect Olfaction Model} example\+: 
\begin{DoxyCode}
tools/\hyperlink{userproject_2MBody__userdef__project_2README_8txt_a320a215d1e27b4de394be70e90d22863}{generate\_run} [\hyperlink{README_8txt_a74a069e3c75797de2636c4dd14daa147}{CPU}/\hyperlink{modelSpec_8h_a39cb9803524b6f3b783344b2f89867b4}{GPU}] [#AL] [#KC] [#LH] [#DN] [gscale] [DIR] [EXE] [MODEL] [DEBUG 
      OFF/ON]. 
\end{DoxyCode}
 To compile {\ttfamily generate\+\_\+run.\+cc}, navigate to the {\ttfamily userproject/\+M\+Body1\+\_\+project} directory and type 
\begin{DoxyCode}
make all
\end{DoxyCode}
 This will generate an executable that you can invoke with, e.\+g., 
\begin{DoxyCode}
./\hyperlink{userproject_2MBody__userdef__project_2README_8txt_a320a215d1e27b4de394be70e90d22863}{generate\_run} 1 100 1000 20 100 0.0025 test1 MBody1 0 
\end{DoxyCode}
 which would generate and simulate a model of the locust olfactory system with 100 projection neurons, 1000 Kenyon cells, 20 lateral horn interneurons and 100 output neurons n the mushroom body lobes.

The tool generate\+\_\+run will generate connectivity matrices for the model {\ttfamily M\+Body1} and store them into files, compile and run the model on the G\+P\+U using these files as inputs and output the resulting spiking activity. To fix the G\+P\+U used, replace the first argument {\ttfamily 1} with the device number of the desired G\+P\+U plus 2, e.\+g., {\ttfamily 2} for G\+P\+U 0. All input and output files will be prefixed with {\ttfamily test1} and will be created in a sub-\/directory with the name {\ttfamily test1\+\_\+output}. The last parameter {\ttfamily 0} will switch the debugging mode off, {\ttfamily 1} would switch it on. More about debugging in the \hyperlink{}{debugging section }.

The M\+Body1 example is already a highly integrated example that showcases many of the features of Ge\+N\+N and how to program the user-\/side code for a Ge\+N\+N application. \hyperlink{Manual}{More details in the User Manual }\hypertarget{Quickstart_page_windows_quick}{}\subsection{Windows Quickstart}\label{Quickstart_page_windows_quick}
All interaction with Ge\+N\+N programs are command-\/line based and hence are executed within a {\ttfamily cmd} window. Open a {\ttfamily cmd} window and navigate to the {\ttfamily userprojects\textbackslash{}tools} directory. 
\begin{DoxyCode}
cd %GENN\_PATH%\(\backslash\)userprojects\(\backslash\)tools
\end{DoxyCode}
 Then type 
\begin{DoxyCode}
make.bat all
\end{DoxyCode}
 to compile a numberof tools that are used by the example projects to generate connectivity and inputs to model networks.

The navigate to the {\ttfamily M\+Body1\+\_\+project} directory. 
\begin{DoxyCode}
cd ..\(\backslash\)MBody1\_project
\end{DoxyCode}
 By typing 
\begin{DoxyCode}
make.bat all
\end{DoxyCode}
 you can compile the {\ttfamily generate\+\_\+run} engine that allows to run a \hyperlink{Examples_ex_mbody}{Insect Olfaction Model} model of the insect mushroom body\+: 
\begin{DoxyCode}
tools/\hyperlink{userproject_2MBody__userdef__project_2README_8txt_a320a215d1e27b4de394be70e90d22863}{generate\_run} [\hyperlink{README_8txt_a74a069e3c75797de2636c4dd14daa147}{CPU}/\hyperlink{modelSpec_8h_a39cb9803524b6f3b783344b2f89867b4}{GPU}] [#AL] [#KC] [#LH] [#DN] [gscale] [DIR] [EXE] [MODEL] [DEBUG 
      OFF/ON]. 
\end{DoxyCode}
 To invoke {\ttfamily generate\+\_\+run.\+exe} type, e.\+g., 
\begin{DoxyCode}
\hyperlink{userproject_2MBody__userdef__project_2README_8txt_a320a215d1e27b4de394be70e90d22863}{generate\_run}.exe 1 100 1000 20 100 0.0025 test1 MBody1 0 
\end{DoxyCode}
 which would generate and simulate a model of the locust olfactory system with 100 projection neurons, 1000 Kenyon cells, 20 lateral horn interneurons and 100 output neurons n the mushroom body lobes.

The tool {\ttfamily generate\+\_\+run.\+exe} will generate connectivity matrices for the model {\ttfamily M\+Body1} and store them into files, compile and run the model on an automatically chosen G\+P\+U using these files as inputs and output the resulting spiking activity. To fix the G\+P\+U used, replace the first argument {\ttfamily 1} with the device number of the desired G\+P\+U plus 2, e.\+g., {\ttfamily 2} for G\+P\+U 0. All input and output files will be prefixed with {\ttfamily test1} and will be created in a sub-\/directory with the name {\ttfamily test1\+\_\+output}. The last parameter {\ttfamily 0} will switch the debugging mode off, {\ttfamily 1} would switch it on. More about debugging in the \hyperlink{}{debugging section }.

The M\+Body1 example is already a highly integrated example that showcases many of the features of Ge\+N\+N and how to program the user-\/side code for a Ge\+N\+N application. \hyperlink{Manual}{More details in the User Manual }\hypertarget{Quickstart_page_how_to}{}\section{How to use Ge\+N\+N}\label{Quickstart_page_how_to}
The most common way to use Ge\+N\+N is to create or modify a program such as {\ttfamily \hyperlink{userproject_2MBody1__project_2generate__run_8cc}{userproject/\+M\+Body1\+\_\+project/generate\+\_\+run.\+cc}}.

In more detail, what {\ttfamily generate\+\_\+run} and similar programs do is\+:
\begin{DoxyEnumerate}
\item use other tools to generate connectivity matrices and store them into files.
\item build the source code of a model simulation. In the example of the M\+Body1\+\_\+project this entails writing neuron numbers into {\ttfamily \hyperlink{sizes_8h}{userproject/include/sizes.\+h}}, and executing {\ttfamily buildmodel.\+sh M\+Body1 \mbox{[}D\+E\+B\+U\+G O\+F\+F/\+O\+N\mbox{]}}. The {\ttfamily buildmodel.\+sh} script compiles the installed code generator in conjunction with the model description, in this example {\ttfamily model/\+M\+Body1.\+cc}, and executes the code generator to generate the complete model simulation code for the M\+Body1 model.
\item compile the generated model code, that can be found in {\ttfamily model/\+M\+Body1\+\_\+\+C\+O\+D\+E/} by invoking {\ttfamily make clean \&\& make} in the {\ttfamily model} directory. It is at this stage that Ge\+N\+N generated model simulation code is combined with user-\/side run-\/time code, in this example {\ttfamily classol\+\_\+sim.\+cu} (classify-\/olfaction-\/simulation) which uses the {\ttfamily map\+\_\+classol} class.
\item finally run the resulting stand-\/alone simulator executable, in the M\+Body1 example {\ttfamily classol\+\_\+sim} in the {\ttfamily model} directory.
\end{DoxyEnumerate}

The {\ttfamily generate\+\_\+run} tool is only a suggested usage scenario of Ge\+N\+N. Alternatively, users can manually execute the four steps above or integrate Ge\+N\+N with development environments of their own choice.

\begin{DoxyNote}{Note}
The usage scenario described was made explicit for Unix environments. In Windows the setup is essentially the same except for the usual operating system dependent syntax differences, e.\+g. the build script is named buildmodel.\+bat, compilation of the generated model simulator would be {\ttfamily make.\+bat clean \&\& make.\+bat all}, or, {\ttfamily nmake /f W\+I\+Nmakefile clean \&\& nmake /f W\+I\+Nmakefile all}, and the resulting executable would be named {\ttfamily classol\+\_\+sim.\+exe}.
\end{DoxyNote}
\hypertarget{Quickstart_page_exs}{}\section{Example projects}\label{Quickstart_page_exs}
Ge\+N\+N comes with several example projects which show how to use its features. The M\+Body1 example discussed above is one of the many provided examples that are described in more detail in \hyperlink{Examples}{Example projects}.\hypertarget{Quickstart_page_ownmodel}{}\section{Defining your own model}\label{Quickstart_page_ownmodel}
If one was to use the library for G\+P\+U code generation only, the following would be done\+:

a) The model in question is defined in a file, say \char`\"{}\+Model1.\+cc\char`\"{}.

b) this file needs to 
\begin{DoxyEnumerate}
\item define \char`\"{}\+D\+T\char`\"{} 
\item include \char`\"{}model\+Spec.\+h\char`\"{} and \char`\"{}model\+Spec.\+cc\char`\"{} 
\item define the values for initial variables and parameters for neuron and synapse populations 
\item contain the model's definition in the form of a function 
\begin{DoxyCode}
\textcolor{keywordtype}{void} \hyperlink{tmp_2model_2MBody__userdef_8cc_a9aeaa0a22980484b2c472564fc9f686e}{modelDefinition}(\hyperlink{classNNmodel}{NNmodel} &\hyperlink{README_8txt_a69fd801b7213948c12d9dd7eebb3ed14}{model}); 
\end{DoxyCode}
 \char`\"{}\+M\+Body1.\+cc\char`\"{} shows a typical example. 
\end{DoxyEnumerate}c) The programmer defines his/her own modeling code along similar lines as \char`\"{}map\+\_\+classol.$\ast$\char`\"{} together with \char`\"{}classol\+\_\+sim.$\ast$\char`\"{}. In this code,


\begin{DoxyItemize}
\item she defines the connectivity matrices between neuron groups. (In the example here those are read from files).
\item she defines input patterns (e.\+g. for Poisson neurons like in the example)
\item she uses \char`\"{}step\+Time\+G\+P\+U(x, y, z);\char`\"{} to run one time step on the G\+P\+U or \char`\"{}step\+Time\+C\+P\+U(x, y, z);\char`\"{} to run one on the C\+P\+U. (both versions are always compiled). However, mixing the two does not make too much sense. The host version uses the same memory whereto results from the G\+P\+U version are copied (see next point)
\item she uses functions like \char`\"{}copy\+State\+From\+Device();\char`\"{} etc to obtain results from G\+P\+U calculations. 
\end{DoxyItemize}