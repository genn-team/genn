Version 2.\+0 of Ge\+N\+N comes with a lot of improvements and added features, some of which have necessitated some changes to the structure of parameter arrays among others.\hypertarget{ReleaseNotes_UserChange}{}\section{User Side Changes}\label{ReleaseNotes_UserChange}

\begin{DoxyEnumerate}
\item Users are now required to call \char`\"{}init\+Ge\+N\+N()\char`\"{} in the model definition function before adding any populations to the neuronal network model.
\item glbscnt is now call glb\+Spk\+Cnt for consistency with glb\+Spk\+Evnt\+Cnt.
\item There is no longer a privileged parameter {\ttfamily Epre}. Spike type events are now defined by a code string {\ttfamily spk\+Evnt\+Threshold} like proper spikes. The only difference ism that Spike type events are specific to a synapse type rather than a neuron type.
\item The function set\+Synapse\+G has been deprecated. In a {\ttfamily G\+L\+O\+B\+A\+L\+G} scenario, the variables of a synapse group are set to the initial values provided in modeldefinition.
\item Due to the split of synaptic models into \hyperlink{classweightUpdateModel}{weight\+Update\+Model} and \hyperlink{structpostSynModel}{post\+Syn\+Model}, the parameter arrays used during model definition need to be carefully split as well so that each side gets the right parameters. For example, previously 
\begin{DoxyCode}
\textcolor{keywordtype}{float} \hyperlink{tmp_2model_2MBody__userdef_8cc_aae91814eee9533981fe819ddae4671c2}{myPNKC\_p}[3]= \{
  0.0,           \textcolor{comment}{// 0 - Erev: Reversal potential}
  -20.0,         \textcolor{comment}{// 1 - Epre: Presynaptic threshold potential}
  1.0            \textcolor{comment}{// 2 - tau\_S: decay time constant for S [ms]}
\};
\end{DoxyCode}
 would define the parameter array of three parameters, {\ttfamily Erev}, {\ttfamily Epre}, and {\ttfamily tau\+\_\+\+S} for a synapse of type {\ttfamily N\+S\+Y\+N\+A\+P\+S\+E}. This now needs to be \char`\"{}split\char`\"{} into 
\begin{DoxyCode}
\textcolor{keywordtype}{float} *myPNKC\_p= NULL;
\textcolor{keywordtype}{float} \hyperlink{tmp_2model_2MBody__userdef_8cc_a0186f2a893745a3325fb929335e73621}{postExpPNKC}[2]=\{
  1.0,            \textcolor{comment}{// 0 - tau\_S: decay time constant for S [ms]}
  0.0             \textcolor{comment}{// 1 - Erev: Reversal potential}
\};
\end{DoxyCode}
 i.\+e. parameters {\ttfamily Erev} and {\ttfamily tau\+\_\+\+S} are moved to the post-\/synaptic model and its parameter array of two parameters. {\ttfamily Epre} is discontinued as a parameter for {\ttfamily N\+S\+Y\+N\+A\+P\+S\+E}. As a consequence the weightupdate model of {\ttfamily N\+S\+Y\+N\+A\+P\+S\+E} has no parameters and one can pass {\ttfamily N\+U\+L\+L} for the parameter array in {\ttfamily add\+Synapse\+Population}. ~\newline
 The correct parameter lists for all defined neuron and synapse model types are listed in the \hyperlink{UserManual}{User Manual}. \begin{DoxyNote}{Note}
If the parameters are not redefined appropriately this will lead to uncontrolled behaviour of models and likely to sgementation faults and crashes.
\end{DoxyNote}

\item Advanced users can now define variables as type \char`\"{}scalar\char`\"{} when introducing new neuron or synapse types. This will at the code generation stage be translated to the model's floating point type (ftype), {\ttfamily float} or {\ttfamily double}. This works for defining variables as well as in all code snippets. Users can also use the expressions \char`\"{}\+S\+C\+A\+L\+A\+R\+\_\+\+M\+A\+X\char`\"{} and \char`\"{}\+S\+C\+A\+L\+A\+R\+\_\+\+M\+I\+N\char`\"{} for \char`\"{}\+F\+L\+T\+\_\+\+M\+I\+N\char`\"{}, \char`\"{}\+F\+L\+T\+\_\+\+M\+A\+X\char`\"{}, \char`\"{}\+D\+B\+L\+\_\+\+M\+I\+N\char`\"{} and \char`\"{}\+D\+B\+L\+\_\+\+M\+A\+X\char`\"{}, respectively. Corresponding definitions of {\ttfamily scalar}, {\ttfamily S\+C\+A\+L\+A\+R\+\_\+\+M\+I\+N} and {\ttfamily S\+C\+A\+L\+A\+R\+\_\+\+M\+A\+X} are also available for user-\/side code whenever the code-\/generated file {\ttfamily runner.\+cc} has been included.
\item The example projects have been re-\/organized so that wrapper scripts of the {\ttfamily generate\+\_\+run} type are now all located together with the models they run instead of in a common {\ttfamily tools} directory. Generally the structure now is that each example project contains the wrapper script {\ttfamily generate\+\_\+run} and a {\ttfamily model} subdirectory which contains the model description file and the user side code complete with Makefiles for Unix and Windows operating systems. The generated code will be deposited in the {\ttfamily model} subdirectory in its own {\ttfamily modelname\+\_\+\+C\+O\+D\+E} folder. Simulation results will always be deposited in a new sub-\/folder of the main project directory.
\item The {\ttfamily add\+Synapse\+Population(...)} function has now more mandatory parameters relating to the introduction of separate weightupdate models (pre-\/synaptic models) and postynaptic models. The correct syntax for the {\ttfamily add\+Synapse\+Population(...)} can be found with detailed explanations in teh \hyperlink{UserManual}{User Manual}.
\item We have introduced a simple performance profiling method that users can employ to get an overview over the differential use of time by different kernels. To enable the timers in Ge\+N\+N generated code, one needs to declare 
\begin{DoxyCode}
networkmodel.setTiming(\hyperlink{modelSpec_8h_aa8cecfc5c5c054d2875c03e77b7be15d}{TRUE});
\end{DoxyCode}
 This will make available and operate G\+P\+U-\/side cude\+Event based timers whose cumulative value can be found in the double precision variables {\ttfamily neuron\+\_\+tme}, {\ttfamily synapse\+\_\+tme} and {\ttfamily learning\+\_\+tme}. They measure the accumulated time that has been spent calculating the neuron kernel, synapse kernel and learning kernel, respectively. C\+P\+U-\/side timers for the simulation functions are also available and their cumulative values can be obtained through 
\begin{DoxyCode}
\textcolor{keywordtype}{float} x= sdkGetTimerValue(&neuron\_timer);
\textcolor{keywordtype}{float} y= sdkGetTimerValue(&synapse\_timer);
\textcolor{keywordtype}{float} z= sdkGetTimerValue(&learning\_timer);
\end{DoxyCode}
 The \hyperlink{Examples_ex_mbody}{Insect Olfaction Model} example shows how these can be used in the user-\/side code. To enable timing profiling in this example, simply enable it for Ge\+N\+N\+: 
\begin{DoxyCode}
model.setTiming(\hyperlink{modelSpec_8h_aa8cecfc5c5c054d2875c03e77b7be15d}{TRUE});
\end{DoxyCode}
 in {\ttfamily M\+Body1.\+cc}'s {\ttfamily model\+Definition} function and define the macro {\ttfamily T\+I\+M\+I\+N\+G} in {\ttfamily classol\+\_\+sim.\+h} 
\begin{DoxyCode}
\textcolor{preprocessor}{#define TIMING}
\end{DoxyCode}
 This will have the effect that timing information is output into {\ttfamily O\+U\+T\+N\+A\+M\+E\+\_\+output/\+O\+U\+T\+N\+A\+M\+E.\+timingprofile}.
\end{DoxyEnumerate}\hypertarget{ReleaseNotes_developerChange}{}\section{Developer Side Changes}\label{ReleaseNotes_developerChange}

\begin{DoxyEnumerate}
\item {\ttfamily allocate\+Sparse\+Arrays()} has been changed to take the number of connections, conn\+N, as an argument rather than expecting it to have been set in the Connetion struct before the function is called as was the arrangement previously.
\item For the case of sparse connectivity, there is now a reverse mapping implemented with revers index arrays and a remap array that points to the original positions of variable values in teh forward array. By this mechanism, revers lookups from post to pre synaptic indices are possible but value changes in the sparse array values do only need to be done once.
\item Spk\+Evnt code is no longer generated whenever it is not actually used. That is also true on a somewhat finer granularity where variable queues for synapse delays are only maintained if the corresponding variables are used in synaptic code. True spikes on the other hand are always detected in case the user is interested in them.
\end{DoxyEnumerate}

~\newline
 

 \hyperlink{Examples}{Previous} $\vert$ \hyperlink{ReleaseNotes}{Top} $\vert$ \hyperlink{UserManual}{Next} 