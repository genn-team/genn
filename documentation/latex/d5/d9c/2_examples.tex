At the moment, there are 5 example projects provided with Ge\+N\+N.\hypertarget{2_examples_ex_mbody}{}\section{Insect Olfaction Model}\label{2_examples_ex_mbody}
Navigate to userproject/\+M\+Body1\+\_\+project and type \char`\"{}../../tools/generate\+\_\+run\char`\"{}

This will show you the command line parameters needed\+: 
\begin{DoxyCode}
tools/\hyperlink{userproject_2MBody__userdef__project_2README_8txt_a320a215d1e27b4de394be70e90d22863}{generate\_run} [\hyperlink{README_8txt_a74a069e3c75797de2636c4dd14daa147}{CPU}/\hyperlink{modelSpec_8h_a39cb9803524b6f3b783344b2f89867b4}{GPU}] [#AL] [#KC] [#LHI] [#DN] [gscale] [DIR] [EXE] [MODEL] [DEBUG
       OFF/ON]
\end{DoxyCode}
 The tool generate\+\_\+run will generate connectivity files for the model M\+Body1, compile this model for the G\+P\+U and execute it. The command line parameters are the numbers of neurons in the different neuropils of the model and an overall synaptic strength scaling factor. A typical call would for example be 
\begin{DoxyCode}
../../tools/\hyperlink{userproject_2MBody__userdef__project_2README_8txt_a320a215d1e27b4de394be70e90d22863}{generate\_run} 1 100 1000 20 100 0.00117 outname classol\_sim MBody1 0
\end{DoxyCode}
 which would generate a model, and run it on the G\+P\+U, with 100 antennal lobe neurons, 1000 mushroom body Kenyon cells, 20 lateral horn interneurons and 100 mushroom body output neurons. All output files will be prefixed with \char`\"{}outname\char`\"{}.

For more about this example model see \cite{nowotny2005self}.\hypertarget{2_examples_ex_poissonizh}{}\section{Izhikevich network driven by Poisson input spike trains\+:}\label{2_examples_ex_poissonizh}
Can be used as\+: 
\begin{DoxyCode}
tools/generate\_run\_PoissonIzh [\hyperlink{README_8txt_a74a069e3c75797de2636c4dd14daa147}{CPU}/\hyperlink{modelSpec_8h_a39cb9803524b6f3b783344b2f89867b4}{GPU}] [#Poisson] [#Izhikevich] [pConn] [gscale] [DIR] [EXE] [MODEL]
       [DEBUG OFF/ON]
\end{DoxyCode}
 Navigate to the \char`\"{}userproject/\+Poisson\+Izh\+\_\+project\char`\"{} directory and type 
\begin{DoxyCode}
../../tools/generate\_run\_PoissonIzh 1 100 10 0.5 2 Outdir PoissonIzh\_sim PoissonIzh 0
\end{DoxyCode}
 This will generate a network of 100 Poisson neurons connected to 10 Izhikevich neurons with a 0.\+5 probability. The same network with sparse connectivity can be used by addind the synapse population with sparse connectivity in \hyperlink{PoissonIzh_8cc}{Poisson\+Izh.\+cc} and by uncommenting the lines following the \char`\"{}//\+S\+P\+A\+R\+S\+E C\+O\+N\+N\+E\+C\+T\+I\+V\+I\+T\+Y\char`\"{} tag in Poisson\+Izh.\+cu.\hypertarget{2_examples_Ex_OneComp}{}\section{Single compartment Izhikevich neuron(s)}\label{2_examples_Ex_OneComp}
This is a minimal example, with only one neuron population (with more or more neurons without any synapses). The model may be used with 
\begin{DoxyCode}
tools/generate\_run\_1comp [\hyperlink{README_8txt_a74a069e3c75797de2636c4dd14daa147}{CPU}/\hyperlink{modelSpec_8h_a39cb9803524b6f3b783344b2f89867b4}{GPU}] [#n] [DIR] [EXE] [MODEL] [DEBUG OFF/ON].
\end{DoxyCode}
 This would create one/a set of tonic spiking Izhikevich neuron(s) with no connectivity, receiving a constant identical 4 n\+A input.

To use it, navigate to the \char`\"{}userproject/\+One\+Comp\+\_\+project\char`\"{} directory and type 
\begin{DoxyCode}
../../tools/generate\_run\_1comp 1 1 Outdir OneComp\_sim OneComp 0.
\end{DoxyCode}
\hypertarget{2_examples_ex_izhnetwork}{}\section{Pulse-\/coupled Izhikevich network}\label{2_examples_ex_izhnetwork}
Can be used as\+: 
\begin{DoxyCode}
tools/generate\_run\_1comp generate\_izhikevich\_network\_run [\hyperlink{README_8txt_a74a069e3c75797de2636c4dd14daa147}{CPU}/\hyperlink{modelSpec_8h_a39cb9803524b6f3b783344b2f89867b4}{GPU}] [#n] [#Conn] [gscale] [outdir] [
      executable name] [\hyperlink{README_8txt_a69fd801b7213948c12d9dd7eebb3ed14}{model} name] [debug OFF/ON] [\hyperlink{README_8txt_acf386c48a14a2099c9220d6bcde40fc8}{use} previous connectivity OFF/ON]
\end{DoxyCode}
 This example creates a pulse-\/coupled network \cite{izhikevich2003simple} with 80\% excitatory 20\% inhibitory neurons, each connecting to \#\+Conn neurons with sparse connectivity.

To use an example, navigate to the \char`\"{}userproject/\+Izh\+\_\+\+Sparse\+\_\+project\char`\"{} directory and type 
\begin{DoxyCode}
../../tools/generate\_izhikevich\_network\_run 1 10000 1000 1 Outdir Izh\_sim\_sparse Izh\_sparse 0 0
\end{DoxyCode}
\hypertarget{2_examples_ex_izhdelay}{}\section{Izhikevich network with delayed synapses}\label{2_examples_ex_izhdelay}
This example project demonstrates the delayed synapse feature of Ge\+N\+N. It creates a network of three Izhikevich neuron groups, connected all-\/to-\/all with fast, medium and slow synapse groups. Neurons in the output group only spike if they are simultaneously innervated by the input neurons, via slow synapses, and the interneurons, via faster synapses.

To run this example project, cd to \char`\"{}\$\+G\+E\+N\+N\+\_\+\+P\+A\+T\+H/userproject/\+Syn\+Delay\+\_\+project\char`\"{} and type 
\begin{DoxyCode}
buildmodel \hyperlink{classSynDelay}{SynDelay} && make clean && make && ./bin/release/syn\_delay 1 output
\end{DoxyCode}
 