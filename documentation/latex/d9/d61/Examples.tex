Ge\+N\+N comes with a number of complete examples. At the moment, there are seven such example projects provided with Ge\+N\+N.

~\newline
\hypertarget{Examples_Ex_OneComp}{}\section{Single compartment Izhikevich neuron(s)}\label{Examples_Ex_OneComp}
This is a minimal example, with only one neuron population (with more or less neurons depending on the command line, but without any synapses). The neurons are Izhikevich neurons \cite{izhikevich2003simple} with homogeneous parameters across the neuron population. The model can be used by navigating to the {\ttfamily userproject/\+One\+Comp\+\_\+project} directory and entering a command line 
\begin{DoxyCode}
./generate\_run <\hyperlink{modelSpec_8h_ad703205f9a4d4bb6af9c25257c23ce6d}{CPU}/\hyperlink{modelSpec_8h_a39cb9803524b6f3b783344b2f89867b4}{GPU}> <n> <OUTNAME> <MODEL> <DEBUG> <FTYPE> <REUSE>.
\end{DoxyCode}
 All parameters are mandatory and signify\+: \begin{DoxyItemize}
\item {\ttfamily C\+P\+U/\+G\+P\+U}\+: Choose whether to run the simulation on C\+P\+U ({\ttfamily 0}) or G\+P\+U ({\ttfamily 1}). \item {\ttfamily n}\+: Number of neurons \item {\ttfamily O\+U\+T\+N\+A\+M\+E}\+: The base name of the output location and output files \item {\ttfamily M\+O\+D\+E\+L}\+: The name of the model to execute, as provided this would be {\ttfamily One\+Comp} \item {\ttfamily D\+E\+B\+U\+G}\+: Whether to start in debug mode (1) or normally (0) \item {\ttfamily F\+T\+Y\+P\+E}\+: Floating point type {\ttfamily F\+L\+O\+A\+T} or {\ttfamily D\+O\+U\+B\+L\+E} \item {\ttfamily R\+E\+U\+S\+E}\+: whether to re-\/use input files\end{DoxyItemize}
This would create {\ttfamily n} tonic spiking Izhikevich neuron(s) with no connectivity, receiving a constant, identical 4 n\+A input current.

For example, navigate to the {\ttfamily userproject/\+One\+Comp\+\_\+project} directory and type 
\begin{DoxyCode}
make all
./generate\_run 1 1 Outdir OneComp\_sim OneComp 0 \hyperlink{modelSpec_8h_ae8690abbffa85934d64d545920e2b108}{FLOAT} 0
\end{DoxyCode}
 to model a single neuron which output will be saved in the {\ttfamily Outdir\+\_\+output} directory.

~\newline
\hypertarget{Examples_ex_poissonizh}{}\section{Izhikevich Network Driven by Poisson Input Spike Trains\+:}\label{Examples_ex_poissonizh}
In this example project there is again a pool of non-\/connected Izhikevich model neurons \cite{izhikevich2003simple} that are connected to a pool of Poisson input neurons with a fixed probability.

The model can be compiled by navigating to the {\ttfamily userproject\textbackslash{}Poisson\+Izh\+\_\+project} directory and typing 
\begin{DoxyCode}
make all
\end{DoxyCode}
 Subsequently it can be invoked using the following command line 
\begin{DoxyCode}
./generate\_run <\hyperlink{modelSpec_8h_ad703205f9a4d4bb6af9c25257c23ce6d}{CPU}/\hyperlink{modelSpec_8h_a39cb9803524b6f3b783344b2f89867b4}{GPU}> <#POISSON> <#\hyperlink{modelSpec_8h_a57aaf01a1c5e075bc2ad849b9b48db31}{IZHIKEVICH}> <PCONN> <GSCALE> <OUTNAME> <MODEL> <DEBUG
       OFF/ON> <FTYPE> <REUSE>
\end{DoxyCode}
 All parameters are mandatory and signify\+: \begin{DoxyItemize}
\item {\ttfamily C\+P\+U/\+G\+P\+U}\+: Choose whether to run the simulation on C\+P\+U ({\ttfamily 0}) or G\+P\+U ({\ttfamily 1}). \item {\ttfamily \#\+P\+O\+I\+S\+S\+O\+N}\+: Number of Poisson input neurons \item {\ttfamily \#I\+Z\+H\+I\+K\+E\+V\+I\+C\+H}\+: Number of Izhikevich neurons \item {\ttfamily P\+C\+O\+N\+N}\+: Probability of connections \item {\ttfamily O\+U\+T\+N\+A\+M\+E}\+: The base name of the output location and output files \item {\ttfamily M\+O\+D\+E\+L}\+: The name of the model to execute, as provided this would be {\ttfamily Poisson\+Izh} \item {\ttfamily D\+E\+B\+U\+G}\+: Whether to start in debug mode (1) or normally (0) \item {\ttfamily F\+T\+Y\+P\+E}\+: Floating point type {\ttfamily F\+L\+O\+A\+T} or {\ttfamily D\+O\+U\+B\+L\+E} \item {\ttfamily R\+E\+U\+S\+E}\+: whether to re-\/use input files\end{DoxyItemize}
For example, navigate to the {\ttfamily userproject/\+Poisson\+Izh\+\_\+project} directory and type 
\begin{DoxyCode}
./generate\_run 1 100 10 0.5 2 Outdir PoissonIzh 0 \hyperlink{modelSpec_8h_a8747af38b86aa2bbcda2f1b1aa0888c2}{DOUBLE} 0
\end{DoxyCode}
 This will generate a network of 100 Poisson neurons connected to 10 Izhikevich neurons with a 0.\+5 probability. The same example network can be used with sparse connectivity (i.\+e. sparse matrix representations for the connectivity within Ge\+N\+N) by using the keyword {\ttfamily S\+P\+A\+R\+S\+E} in the add\+Synapse\+Population instead of {\ttfamily D\+E\+N\+S\+E} in \hyperlink{PoissonIzh_8cc}{Poisson\+Izh.\+cc} and by uncommenting the lines following the comment {\ttfamily //\+S\+P\+A\+R\+S\+E C\+O\+N\+N\+E\+C\+T\+I\+V\+I\+T\+Y} in the file Poisson\+Izh.\+cu. In this example the model would be simulated with double precision variables and input files and connectivity would not be reused from earlier runs.

~\newline
\hypertarget{Examples_ex_izhnetwork}{}\section{Pulse-\/coupled Izhikevich Network}\label{Examples_ex_izhnetwork}
This example model is inspired by simple thalamo-\/cortical network of Izhikevich \cite{izhikevich2003simple} with an excitatory and an inhibitory population of spiking neurons that are randomly connected.

The model can be built by navigating to the {\ttfamily userproject/\+Izh\+\_\+\+Sparse\+\_\+project} directory and typing 
\begin{DoxyCode}
make all
\end{DoxyCode}
 It can then be used as 
\begin{DoxyCode}
./generate\_run <\hyperlink{modelSpec_8h_ad703205f9a4d4bb6af9c25257c23ce6d}{CPU}/\hyperlink{modelSpec_8h_a39cb9803524b6f3b783344b2f89867b4}{GPU}> <#N> <#CONN> <GSCALE> <OUTNAME> <MODEL> <DEBUG OFF/ON> 
 <FTYPE> <REUSE>
\end{DoxyCode}
 All parameters are mandatory and signify\+: \begin{DoxyItemize}
\item {\ttfamily C\+P\+U/\+G\+P\+U}\+: Choose whether to run the simulation on C\+P\+U ({\ttfamily 0}) or G\+P\+U ({\ttfamily 1}). \item {\ttfamily \#\+N}\+: Number of neurons \item {\ttfamily \#\+C\+O\+N\+N}\+: Number of connections per neuron \item `\+G\+S\+C\+A\+L\+E'\+: General scaling of synaptic conductances \item {\ttfamily O\+U\+T\+N\+A\+M\+E}\+: The base name of the output location and output files \item {\ttfamily M\+O\+D\+E\+L}\+: The name of the model to execute, as provided this would be {\ttfamily Izh\+\_\+sparse} \item {\ttfamily D\+E\+B\+U\+G}\+: Whether to start in debug mode (1) or normally (0) \item {\ttfamily F\+T\+Y\+P\+E}\+: Floating point type {\ttfamily F\+L\+O\+A\+T} or {\ttfamily D\+O\+U\+B\+L\+E} \item {\ttfamily R\+E\+U\+S\+E}\+: whether to re-\/use input files\end{DoxyItemize}
The model creates a pulse-\/coupled network \cite{izhikevich2003simple} with 80\% excitatory 20\% inhibitory neurons, each connecting to \#\+C\+O\+N\+N neurons using the sparse matrix connectivity methods of Ge\+N\+N.

For example, typing 
\begin{DoxyCode}
./tools/generate\_izhikevich\_network\_run 1 10000 1000 1 Outdir Izh\_sparse 0 \hyperlink{modelSpec_8h_ae8690abbffa85934d64d545920e2b108}{FLOAT} 0 
\end{DoxyCode}
 generates a random network of 8000 excitatory and 2000 inhibitory neurons which each have 1000 outgoing synapses to randomly chosen post-\/synaptic target neurons. The synapses are of a simple pulse-\/coupling type. The results of the simulation are saved in the directory {\ttfamily Outdir\+\_\+output}, debugging is switched off, and the connectivity is generated afresh (rather than being read from existing files). \begin{DoxyNote}{Note}
If connectivity were to be read from files, the connectivity files would have to be in the {\ttfamily inputfiles} sub-\/directory and be named according to the names of the synapse populations involved, e.\+g., {\ttfamily g\+Izh\+\_\+sparse\+\_\+ee} ($<$variable name$>$=\char`\"{}\char`\"{}$>$={\ttfamily g} $<$model name$>$=\char`\"{}\char`\"{}$>$={\ttfamily Izh\+\_\+sparse} \+\_\+$<$synapse population$>$={\ttfamily \+\_\+ee}). These name conventions are not part of the core Ge\+N\+N definitions and it is the privilege (or burden) of the user to find their own in their own versions of {\ttfamily generate\+\_\+\+X\+X\+X\+\_\+run}.
\end{DoxyNote}
~\newline
\hypertarget{Examples_ex_izhdelay}{}\section{Izhikevich network with delayed synapses}\label{Examples_ex_izhdelay}
This example project demonstrates the feature of synaptic delays in Ge\+N\+N. It creates a network of three Izhikevich neuron groups, connected all-\/to-\/all with short, medium and long delay synapse groups. Neurons in the output group only spike if they are simultaneously innervated by the input neurons, via synapses with long delay, and the interneurons, via synapses with shorter delay.

To run this example project, navigate to {\ttfamily userproject/\+Syn\+Delay\+\_\+project} and type, e.\+g., 
\begin{DoxyCode}
buildmodel \hyperlink{classSynDelay}{SynDelay}
make clean && make 
./bin/release/syn\_delay 1 output
\end{DoxyCode}


~\newline
\hypertarget{Examples_ex_mbody}{}\section{Insect Olfaction Model}\label{Examples_ex_mbody}
This project implements the insect olfaction model by Nowotny et al. \cite{nowotny2005self} that demonstrates self-\/organized clustering of odours in a simulation of the insect antennal lobe and mushroom body. As provided the model works with conductance based Hodgkin-\/\+Huxley neurons \cite{Traub1991} and several different synapse types, conductance based (but pulse-\/coupled) excitatory synapses, graded inhibitory synapses and synapses with a type of S\+T\+D\+P rule.

To explore the model navigate to {\ttfamily userproject/\+M\+Body1\+\_\+project/} and type 
\begin{DoxyCode}
make all
./generate\_run 
\end{DoxyCode}
 This will show you the command line parameters that are needed, 
\begin{DoxyCode}
tools/generate\_run <\hyperlink{modelSpec_8h_ad703205f9a4d4bb6af9c25257c23ce6d}{CPU}/\hyperlink{modelSpec_8h_a39cb9803524b6f3b783344b2f89867b4}{GPU}> <#AL> <#KC> <#LHI> <#DN> <GSCALE> <OUTNAME> <MODEL> <DEBUG> <FTYPE> <
      REUSE>
\end{DoxyCode}
 All parameters are mandatory and signify\+: \begin{DoxyItemize}
\item {\ttfamily C\+P\+U/\+G\+P\+U}\+: Choose whether to run the simulation on C\+P\+U ({\ttfamily 0}) or G\+P\+U ({\ttfamily 1}). \item {\ttfamily \#\+A\+L}\+: Number of neurons in the antennal lobe (A\+L), the input neurons to this model \item {\ttfamily \#\+K\+C}\+: Number of Kenyon cells (K\+C) in the \char`\"{}hidden layer\char`\"{} \item {\ttfamily \#\+L\+H}\+: Number of lateral horn interneurons, implementing gain control \item {\ttfamily \#\+D\+N}\+: Number of decision neurons (D\+N) in the output layer \item {\ttfamily G\+S\+C\+A\+L\+E}\+: A general rescaling factor for snaptic strength \item {\ttfamily O\+U\+T\+N\+A\+M\+E}\+: The base name of the output location and output files \item {\ttfamily M\+O\+D\+E\+L}\+: The name of the model to execute, as provided this would be {\ttfamily M\+Body1} \item {\ttfamily D\+E\+B\+U\+G}\+: Whether to start in debug mode (1) or normally (0) \item {\ttfamily F\+T\+Y\+P\+E}\+: Floating point type {\ttfamily F\+L\+O\+A\+T} or {\ttfamily D\+O\+U\+B\+L\+E} \item {\ttfamily R\+E\+U\+S\+E}\+: whether to re-\/use input files\end{DoxyItemize}
The tool generate\+\_\+run will generate connectivity files for the model {\ttfamily M\+Body1}, compile this model for the C\+P\+U and G\+P\+U and execute it. The command line parameters are the numbers of neurons in the different neuropils of the model and an overall synaptic strength scaling factor. A typical call would be, e.\+g., 
\begin{DoxyCode}
../../tools/generate\_run 1 100 1000 20 100 0.0025 test1 MBody1 0 \hyperlink{modelSpec_8h_ae8690abbffa85934d64d545920e2b108}{FLOAT} 0
\end{DoxyCode}
 which would generate a model, and run it on the G\+P\+U (first parameter), with 100 antennal lobe neurons, 1000 mushroom body Kenyon cells, 20 lateral horn interneurons and 100 mushroom body output neurons. All output files will be prefixed with {\ttfamily test1} and stored in {\ttfamily test1\+\_\+output}. The model that is run is defined in {\ttfamily model/\+M\+Body1.\+cc}, debugging is switched off, the model would be simulated using float (single precision floating point) variables and parameters and the connectivity and input would be generated afresh for this run.

As provided, the model outputs a file {\ttfamily test1.\+out.\+st} that contains the spiking activity observed in the simulation, where there are two columns in this A\+S\+C\+I\+I file, the first one containing the time of a spike and the second one the I\+D of the neuron that spiked. Users of matlab can use the scripts in the {\ttfamily matlab} directory to plot the results of a simulation. For more about the model itself and the scientific insights gained from it see \cite{nowotny2005self}.

~\newline
\hypertarget{Examples_ex_mbody_userdef}{}\section{Insect Olfaction Model with User-\/\+Defined Types}\label{Examples_ex_mbody_userdef}
This examples recapitulates the exact same model as \hyperlink{Examples_ex_mbody}{Insect Olfaction Model} above but with user-\/defined model types for neurons and synapses throughout. It is run the same way as \hyperlink{Examples_ex_mbody}{Insect Olfaction Model}, e.\+g. 
\begin{DoxyCode}
../../tools/generate\_run 1 100 1000 20 100 0.0025 test1 MBody\_userdef 0 \hyperlink{modelSpec_8h_ae8690abbffa85934d64d545920e2b108}{FLOAT} 0
\end{DoxyCode}
 But the way user-\/defined types are used should be very instructive to advanced users wishing to do the same with their models. 

 \hyperlink{Quickstart}{Previous} $\vert$ \hyperlink{Examples}{Top} $\vert$ \hyperlink{ReleaseNotes}{Next} 